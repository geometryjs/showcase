\chapter{Závěr}
\label{chap:conclusion}

Výsledkem této práce je knihovna \textit{GeometryJS} dostupná na \textit{\hyperlink{https://github.com/geometryjs/geometry.js}{GitHubu}}\footnote{\url{https://github.com/geometryjs/geometry.js}} a \textit{\hyperlink{https://www.npmjs.com/package/@jiricekcz/geometry.js}{NPM}}\footnote{\url{https://www.npmjs.com/package/@jiricekcz/geometry.js}} spolu s jejíc dokumentací.
Knihovna dokáže pracovat s body, vektory, přímkami a hodnotami, ale je snadno rozšířitelná o další geometrické objekty.

\section{Porovnání se standardním softwarovým projektem}
\label{sec:comparison}

Při vytváření knihovny si můžeme všimnout mnoha rozdílů oproti vytváření klasického softwarového projektu.

Musí být kladený zvýšený důraz na stabilitu exportovaných struktur, při tom ale musí být zachována flexibilita pro rozšíření.
Také je nutné myslet na to, že knihovna bude používána v mnoha různých prostředích, tudíž je nutné minimalizovat požadavky na prostředí, ve kterém knihovna funguje.
Deskriptivní názvy třííd, vlastností a metod jsou také důležitější, jelikož se v nich bude muset orientovat mnoho uživatelů knihovny.
Je též potřeba dbát na optimalizaci výkonu, protože nemůžeme efektivně předpovídat v jakých prostředíích a jak budou uživatelé knihovnu používat a co za požadavky na výkon na ni budou mít.

Na druhou stranu není třeba vytvářet grafické uživatelské rozhraní a při tvorbě negrafického uživatelského rozhraní můžeme předpokládat alespoň základní znalost programování uživatele.
Pro publikaci a dokumentaci můžeme používat nástroje, které by uživatel-neprogramátor nemusel být schopen použít, ale pro nás jako vývojáře jsou jednodušší a rychlejší.

