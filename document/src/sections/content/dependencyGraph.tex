\chapter{Graf závislostí}
\label{chap:dependency-graph}

Graf závislostí obsahuje informace o vztazích mezi \uv{bound} objekty.
Jedná se o orientovaný necyklický graf\cite{wikipedia:directed-graph}, jehož vrcholy jsou objekty a hrany reprezentují závislosti mezi nimi.
Orientovanost grafu je implikována tím, že se jedná o graf necyklický.

Graf je v paměti uložen tak, že každý objekt obsahuje \texttt{Iterable}\footnote{\texttt{Iterable} je vestavěné rozhraní, které reprezentuje objekt, kterým lze iterovat.} objektů, které na něm závisí, a \texttt{Iterable} objektů, na kterých závisí on\cite[line 17, 21]{geometryjs:source:interfaces:dependencyNode.ts}.
Elementem tohoto grafu je vždy objekt typu \texttt{DependencyNode}.

\section[Přidávání objektů]{Přidávání objektů do grafu}
\label{sec:adding-objects-to-graph}

O přidávání objektů do grafu závislostí se nemusí starat uživatel knihovny.
Objekt je přidán vždy při instanciaci třídy.
Toto je zajištěno pomocí parametru \texttt{dependencies} konstruktoru třídy \texttt{DependencyNode}\cite[line 13-17]{geometryjs:source:geometryObjects:dependencyNode.ts}.
Tento parametry je při dědění tříd vždy předán konstruktoru rodičovské třídy. 
Můžeme tak mít libovolný počet \uv{mezitříd}, dokud se nedostaneme k dost konktrétní třídě, která už má stanoveno, na kterých objektech závisí.

O přidání reference druhého směru (tedy z objektu na kterém závisí nově vytvořený objekt na nově vytvořený objekt) se stará metoda \texttt{registerDependant}\cite[line 32]{geometryjs:source:interfaces:dependencyNode.ts}.

\section[Procházení]{Procházení grafem}
\label{sec:traversing-the-graph}

Procházení grafem je možné libovolným alogritmem pro procházení grafu.
Pokud budeme iterovat přes \texttt{deepDependencies}\cite[line 17]{geometryjs:source:interfaces:dependencyNode.ts} nebo přes \texttt{deepDependants}\cite[line 21]{geometryjs:source:interfaces:dependencyNode.ts}, využíváme algoritmu pro prohledávání do hloubky\cite{wikipedia:depth-first-search}.
Jelikož se jedná o orientovaný graf, nikoli strom, je nutné si pamatovat již navštívené vrcholy, abychom se nezacyklili.
K tomu využíváme vestavěný datový typ \texttt{Set}.

\section[Necykličnost]{Zajištění necykličnosti grafu}
\label{sec:acyclicity}

Protože při výpočtu vlastností jednotlivých objektů budeme rekurzivně počítat potřebné vlastnosti všech objektů, na kterých dané objekty závisí, je nutné zamezit cyklickým závislostem.
Tohoto snadno docílíme tak, že nedovolíme změnit závislosti po vytvoření objektu a budeme všechny závislosti požadovat již při vytváření objektu.
V kódu tohoto docílíme tak, že metodu na přidávání závislostí (metoda \texttt{registerDependency}\cite[line 20-23]{geometryjs:source:geometryObjects:dependencyNode.ts}) uděláme \texttt{protected} a zavoláme ji pouze v konstruktoru\footnote{Je stále teoreticky možné vytvořit cyklickou závislost, ale lze prakticky vyloučit, že bychom ji vytvořili \uv{omylem}, jelikož kód k tomu potřebný je tak nestadardní, že si nelze neuvědomit, že děláme něco špatně.}.