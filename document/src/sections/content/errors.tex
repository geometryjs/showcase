\chapter{Chybové hlášky}
\label{chap:error-messages}

Další nezbynou součástí knihovny jsou chybové hlášky.
Ne vždy uživatel knihovny použije všechny metody správně a obzvláště v jazyce bez striktního statického typování\footnote{Nejčastě nástává problém při použítí JavaScriptu. Může nastat i v TypeScriptu při nadužívaní \texttt{any} typu nebo s použitím například \textit{ts-ignore} a \textit{ts-expect-error} direktiv.}.
Když k takové chybě dojde, je vhodné uživateli poskytnout co nejvíce informací o tom, proč k chybě došlo, popřípadě, pokud chybu očekává, mu umožnit ji vyřešit v kódu.
V TypeScriptu se \uv{error handling} standardně řeší pomocí \texttt{try} a \texttt{catch} bloků a klíčového slova \texttt{throw}.

My rozlišujeme dva typy chyb - fatální a nefatální. Pokud dojde k problému nefatálnímu (například nedefinovaný výraz\footnote{Za nedefinovaný výraz se považuje například $\frac{0}{0}$} ve výpočtu), vrátí daná metoda nebo funkce \texttt{NaN}.
Nefatální chyba se propaguje výpočty a projeví se tedy pouze pokud ovlivní hodnotu, kterou uživatel knihovny používá.

Fatální chyba je chyba, která způsobí, že některá z metod nebo funkcí nemůže pokračovat v práci, nebo je to chyba, která je považována za tak závažnou, že by mohla způsobit obecně chaotické chování knihovny.
Takové chyby jsou řešeny pomocí \texttt{throw} a jednoho z \uv{error typů}, které knihovna poskytuje.
Tyto chyby jsou pak řešeny pomocí \texttt{try} a \texttt{catch} bloků.

Chyby jsou navrženy tak, aby při standardním používání knihovny nebylo nutné použít \texttt{catch} bloky, protože značně zpomalují kód\cite{stackoverflow:try-catch-performance:2013}.