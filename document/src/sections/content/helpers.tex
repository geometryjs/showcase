\chapter{Pomocné struktury}
\label{chap:helpers}

Každý rozsáhlejší softwarový projekt má své pomocné funkce a třídy. 
Jedná se typicky o funkce a třídy využívané na více místech, které ale zároveň nejsou specifické pro daný projekt.
Často to jsou například funkce pro práci s datovými strukturám nebo přímo implementace těchto datových struktur.

\section{Pomocné třídy}
\label{sec:helper-classes}

Jelikož zatím knihovna nepoužívá žádné složitější datové struktury, pomocná třída je pouze jedna\footnote{Exportovány jsou dvě, ale obě jsou velmi podobné a není třeba je rozebírat jednotlivě.}\cite{geometryjs:source:helpers:memoryMapCache.ts}.

\subsection[Cache]{Pomocná třída Cache}
\label{subsec:helper-class-cache}

Jak již bylo zmíněno v \hyperref[chap:dependency-graph]{kapitole \ref*{chap:dependency-graph}}, knihovna využívá graf závislostí.
Když už je tento graf k dispozici a všechny objekty jsou v něm uloženy, je vhodné si výpočty, které již byly provedeny, někde uložit.
Každý objekt implementující rozhraní \texttt{DependencyNode} obsahuje metodu \texttt{update}\cite[line 26]{geometryjs:source:interfaces:dependencyNode.ts}.
Tato metoda je volána vždy, když se změní nějaký objekt, na kterém daný objekt závisí, můžeme ji proto spojit s čištěním cache.

\subsubsection[Rozhraní]{Rozhraní \texttt{Cache}}
\label{subsubsec:helper-class-cache-interface}

Rozhraní \texttt{Cache}\cite{geometryjs:source:interfaces:cache.ts} nám definuje jak by implementacepomocných tříd měla vypadat.
Rozlišuje, zda je cache \uv{NonEmpty}.
Pokud je cache \uv{NonEmpty}, musí všechny jeho prvky být vždy definované.
Pokud cache \uv{NonEmpty} není, mohou jeho prvky být \texttt{undefined}\footnote{TypeScript zajišťuje \uv{null safty} pomocí \uv{type unions} s typem \texttt{undefined}.}.

Rozšířením rozhraní \texttt{Cache} je rozhraní \texttt{IterableCache}\cite[line 49 - 54]{geometryjs:source:interfaces:cache.ts}, které nám zajišťuje, že cache je iterovatelná.

\subsubsection[Implementace]{Implementace rozhraní \texttt{Cache}}
\label{subsubsec:helper-class-cache-implementation}

Implementace rozhraní \texttt{Cache} je třída \texttt{MemoryMapCache}\cite[line 7 - 33]{geometryjs:source:helpers:memoryMapCache.ts} a její rozšíření \texttt{MemoryMapCacheWithAutomaticCalculation}\cite[line 38 - 59]{geometryjs:source:helpers:memoryMapCache.ts}.
Obě tyto třídy jsou implementovány pomocí vestavěného rozhraní \texttt{Map}\cite{mdn:map}.

Třída \texttt{MemoryMapCacheWithAutomaticCalculation} v konstruktoru přijímá funkce pro výpočet hodnoty, pokud není v cache nalezena.
Tím tato třída zajišťuje, že vždy, když se pokusíme získat hodnotu z cache, získáme ji (pokud existuje, tak z cache, pokud neexistuje, tak ji vypočítáme)\cite{geometryjs:wiki:helpers}.

\section{Pomocné funkce}
\label{sec:helper-functions}

Nejčastějším typem pomocných struktur jsou funkce. 

\subsection[Rovnost]{Funkce pro zjišťování rovnosti}
\label{subsec:helper-function-equality}

Úkon, který se na první pohled může zdát triviální, je zjištění rovnosti dvou čísel.
V JavaScriptu existuje pouze jeden typ čísel, a to \texttt{number}.
Specifikace nám nezaručuje, jakým datovým typem bude číslo reprezentováno\footnote{JIT může v některých případech zkompilovat čísla do 32-bitových \uv{integerů}. Zajišťuje nám však, že chování bude ekvivalentní jako při použití 64-bitového \uv{floatu}}, musí být ale schopné dosáhnout přesnosti a rozsahu 64-bitového čísla s plovoucí desetinnou čárkou\cite{mdn:number}.

Porovnávání čísel s plovoucí desetinnou čárkou je však známý problém\cite{wikipedia:floating-point-arithmetic}.
Pokud k výsledku, který by matematicky měl být ekvivalentní, dojdeme dvěma různými cestami, může se stát, že výsledky budou rozdílné\footnote{Rozdlínými výsledky jsou myšleny výsledky s rozdílnou bitovou reprezentací dle standardu \textit{IEEE 754}\cite{wikipeadia:double-precision-floating-point-format}}.
To je způsobeno tím, že mezivýsledky jsou vždy zaokrouhlovány na konečný počet desetinných míst.

Řešením tohoto problému je porovnávání s přesností menší než je maximální přesnost proměnné dle standardu \textit{IEEE 754}. 
Přesnost je nastavována globálně.
Výchozí hodnota je $43$ bitů, to je o $10$ bitů méně než je maximální přesnost proměnné typu \texttt{number} v JavaScriptu.
Toto nastavení je vhodné pro většinu případů.

Ve speciálních případech (například \texttt{NaN}, \texttt{Infinity}, \texttt{-Infinity}) se řídíme standardním chováním operátoru \texttt{===}.\cite{geometryjs:wiki:helpers}

\subsubsection[Rovnost nule]{Funkce pro porovnávání s nulou}
\label{subsubsec:helper-function-compare-to-zero}

Jediným okrajovým případem je porovnávání s nulou.
V tomto případě nelze porovnat pouze s určitou přesností, jelikož absolutní přesnost nuly není známa.
Pro tento případ existuje globální nastavení \texttt{unit}. 
To nám udává číslo, kolem kterého se budou výsledky pohybovat. 
Platí, že číslo $x$ je rovno 0 právě tehdy, když $x + \mathop{unit}$ je rovno $\mathop{unit}$ podle podmínek stanovených pro obecnou rovnost čísel výše.\cite{geometryjs:source:interfaces:float.ts}

\subsection[Název typu]{Získávání názvu typu objektu}
\label{subsec:getting-type-name}

Naše knihovna je napsána v TypeScriptu a tak v dokumentaci i ve zdrojovém kódu pracujeme s typy.
JavaScript však typy nezná\cite{mdn:object-type} a tak pokud dojde k typové chybě, nejsme schopni uživateli popsat, co za typovou chybu udělal.
Abychom ale mohli uživateli alespoň částečně napověděli, co udělal za chybu, vytvořili jsme funkci \texttt{getTypeString}\cite{geometryjs:source:helpers:getTypeString.ts}, která se pokusí odhadnout, jak by vypadal TypeScriptový typ objektu, který ji poskytneme.

\section{Pomocné typy}
\label{sec:helper-types}

TypeScript má velmi silný typový systém, který nám umožňuje vytvářet vlastní typy.
Komplexita typů může být tak vysoká, že by snížila čitelnost kódu, kdyby byly všechny vyvářeny na místech, kde se používají.
Některé typy se také mohou hodit na více místech.
Nejčastěji používané typy, které přímo nesouvisí s náplní knihovny, jsou proto vytvořeny jako pomocné typy.

\subsection[Some]{Typ \texttt{Some}}
\label{subsec:helper-type-some}

Často se nám stane, že chceme typově popsat libovolnou hodnotu, která není \texttt{undefined} nebo \texttt{null}.
Takový typ v TypeScriptu vestavěný není, ale můžeme si ho definovat sami.
Definujeme proto typ \texttt{Some}\cite[line 4]{geometryjs:source:helpers:types:general.ts}, který popisuje všechny hodnoty kromě \texttt{undefined} a \texttt{null}.