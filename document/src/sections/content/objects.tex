\chapter{Objekty}
\label{chapter:objects}

Základním objektem knihovny je \texttt{Plane}\footnote{Pokud se zde zmiňujeme o objektech \texttt{NázevObjektu}, myslíme tím vždy objekt, který implemetuje rozhraní \texttt{NázevObjektu}.}\cite{geometryjs:source:interfaces:plane.ts}. 
Jedná se o rovinu, ve které se budou všechny objekty nacházet.
Zároveň je to objekt, který bude obsahovat metody na vytváření nových objektů\cite{geometryjs:wiki:geometryObjects}.

\section[Kategorizace]{Kategorizace objektů na základě některých vlastností}
\label{sec:object-categorization}

Pro přehlednost a snadnější orientaci v knihovně rozlišuje dokumentace několik kategorí objektů\footnote{Rozdělním objektů je zde myšleno rozdělení tříd nebo rozhraní, nikoli konkrétních instancí těchto tříd nebo tříd implementujích tyto rozhraní.}\cite{geometryjs:wiki:geometryObjects}.

\subsection[ObjectWithType]{\texttt{ObjectWithType}}
\label{subsec:object-with-type}

Do kategorie\footnote{Jedná se zároveň i o rozhraní.} \texttt{ObjectWithType} patří objekty, které reprezentují geometrické objekty. 
Jediným poždavkem na objekty v této kategorii je, aby měly vlastnost \texttt{objectType}, která určuje typ objektu, a vlastnost \texttt{virtual}, která určuje, zda se jedná o virtuální objekt (o tom více v další sekci).
Pokud tedy někde bereme jako parametr objekt typu \texttt{ObjectWithType}, víme, že jsme rychle schopni zjistit, jaký geometrický objekt reprezentuje.\cite[volný překlad]{geometryjs:wiki:geometryObjects}

\subsection[Virtuální a nevirtuální]{Virtuální a nevirtuální objekty}
\label{subsec:virtual-and-non-virtual}

Objekty z kategorie \texttt{ObjectWithType} mohou být virtuální\footnote{Pojem virtuální se v jiných programovacích jazycích využívá k popisu metod, které lze \uv{overridnout} ve třídě, která tuto metodu dědí. V TypeScriptu nejsou žádné virtuální metody (respektive jsou všechny virtuální, a proto to není nutno specifikovat), využíváme tedy tohoto termínu pro jiný účel.} nebo nevirtuální.
Virtuálními objekty se rozumí takovým objektům, které nereprezentují geometrický objekt, který by bylo možné umístít do roviny (například vektor nebo číslo).
Nevirtuální jsou tedy objekty, které reprezentují geometrické objekty, které je možné umístit do roviny (například bod nebo přímka).\cite[volný překlad]{geometryjs:wiki:geometryObjects}

Toto dělení se netýká objektů nespadajících do kategorie \texttt{ObjectWithType}.

\subsection[Bound a Unbound]{\uv{Bound} a \uv{Unbound} objekty}
\label{subsec:bound-and-unbound}

Aby byla knihovna schopna správně reagovat na změnu objektu, musí vědět, které objekty od něj závisí.
Toto je zajištěno promocí Grafu závislostí (viz sekce \ref{chap:dependency-graph}).

Objekty, které označujeme jako \uv{bound} jsou objekty, které jsou součástí grafu závislostí.
Ty, které nejsou součástí grafu závislostí, označujeme jako \uv{unbound}.

\uv{Bound} objekty jsou vytvářeny pomocí metod objektu \texttt{Plane} (viz sekce \ref{sec:creating-objects}) a jsou brány za dlouho žijící objekty. 
Není tedy vhodné je vytvářet v cyklu či například v každém snímku animace.
Jsou optimalizovány pro rychlé přepočítávání vlastností při změně závislých objektů, nikoli pro rychlé vytváření a mazání.
Typicky z nich složíme scénu (to si můžeme představit stejně jako je tomu v GeoGebře).
Musíme si taky při jejich vytváření dávat pozor na to, aby nedošlo k \uv{memory leaku}\footnote{JavaScript je jazyk s \uv{garbage collectorem}\cite{mdn:memory-management}, který se stará o uvolňování paměti. Jedná se o velmi složitý proces, ale lze ho snadno shrnout tak, že pokud na objekt není žádná reference, je uvolněn. Na \uv{bound} objekty je vždy držena reference, tudíž se nikdy neuvolní.} 

\uv{Unbound} objekty slouží především pro přenos dat mezi funkcemi. 
Jejich hodnoty nejsou závislé na žádných jiných objektech a neaktualizují se při změně jiných objektů.
Jejich vytváření nemá přílišný vliv na výkon\footnote{Vliv není větší než při standardní instanciaci objektů v JavaScriptu.}, tudíž je možno je vytvářet v cyklu či v každém snímku animace a používat je k výpočtům.\cite[volný překlad]{geometryjs:wiki:geometryObjects}


\section[Vytváření]{Vytváření objektů pomocí objektu \texttt{Plane}}
\label{sec:creating-objects}

Metody objektu \texttt{Plane} nám umožnují vytvářet nové objekty. 
Tyto metody se (i podle API dokumentace) dělí do několika skupin - metody typu \texttt{create}, \texttt{construct} a \texttt{extract}\footnote{Metodou typu \texttt{typMetody} se v tomto případě chápe metoda jejíž název začíná sekvencí znaků \textit{typMetody}. Zároveň má odpovídající \uv{tag} v API dokumentaci\cite{geometryjs:api}}\cite{geometryjs:wiki:geometryObjects}.
Těmito metodami vytváříme výhradně \uv{bound} objekty. 
\uv{Unbound} objekty vytváříme například metodami jiných tříd, které ovšem nemají vlastní kategorizaci.

\subsection[Create]{Metody typu \texttt{create}}
\label{subsec:create-methods}

Metody typu \texttt{create} slouží k vytváření objektů z jednoho nebo více objektů a/nebo hodnot zjevnou\footnote{Zjevnou cestou se zde rozumí tomu, že pokud známe typy vstupních objektů a typ výstupu, je nám jasné, jakým procesem ze vstupu výstup získáme. Jedná se například o vytvoření přímky ze dvou bodů (je zjevné, že přímku vytvoříme tak, aby oběma body procházela).} cestou\cite{geometryjs:wiki:geometryObjects}.
Názvy metod tohoto typu jsou většinou odvozeny od názvů vstupních objektů a/nebo hodnot a mohou vypadat například takto: 

\texttt{create[TypVýsledku]From[TypParametru1]And[TypParametru2]()}

\subsection[Construct]{Metody typu \texttt{construct}}
\label{subsec:construct-methods}

Metody typu \texttt{construct} slouží k vytváření objektů z jednoho nebo více objektů a/nebo hodnot nezjevnou\footnote{Nezjevnou cestou se rozumí takové konstrukci, u které je nutné k jednoznačnosti procesu přidat i název konstrukce. Jendná se například o vedení kolmice k přímce bodem (pokud bychom zadali pouze přímku a bod, není jednoznačně jasné, jak k výsledné přímce máme přijít).} cestou \cite{geometryjs:wiki:geometryObjects}.
Názvy metod tohoto typu se neřídí striktními pravidli, pouze se snaží být co nejvíce popisné, například:

\texttt{constructPerpendicularLineFromPoint()}

\subsection[Extract]{Metody typu \texttt{extract}}
\label{subsec:extract-methods}

Metody typu \texttt{extract} slouží k \uv{boundnutí} objektů.
Typicky jsou využity v případech, kdy chceme získat nějakou vlastnost objektu, která je běžně vrácena metodou jako \uv{unbound} objekt, jako \uv{bound} objekt\cite{geometryjs:wiki:geometryObjects}.
Názvy metod tohoto typu jsou většinou odvozeny od názvů vstupního objektu a extahoané vlastnosti a mohou vypadat například takto:

\texttt{extract[PopisVýsledku][TypVýsledku]From[TypParametru]()}