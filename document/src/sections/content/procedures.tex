\chapter{Procedury}
\label{chap:procedures}

Knihovna potřebuje (nebo alespoň v budoucnu může potřebovat) velké množství matematických výpočtů.
Abychom se vyhnuli opakování a umožnili lepší segmentaci kódu, abstrahujeme tyto výpočty do tzv. \uv{procedur}\cite{geometryjs:wiki:procedures}.

\section[Definice]{Definice procedury}
\label{sec:procedure-definition}

Procedurou nazveme \uv{pure} funkci, která bere jako parametr jeden JavaScriptový objekt a vrací jeden JavaScriptový objekt\cite{geometryjs:wiki:procedures}.

\section[Implementace]{Implementace procedury}
\label{sec:procedure-implementation}

Procedury budeme implementovat jako třídy implementují rozhraní \texttt{Procedure}\cite{geometryjs:source:interfaces:procedure.ts}.
Pro zjednodušení implementace máme k dispozici třídu \texttt{Procedure}\cite{geometryjs:source:procedures:procedure.ts}.

\subsection{Využití třidy}
\label{subsec:procedure-usage-of-class}

Někteří si možná mohou položit otázku, proč využíváme třídu, když se jedná o pouhou funkci.
Důvodem je, že procedury jsou takto objekty a můžeme jim tedy přidávat další metody nebo vlastnosti.
Zatím této možnosti knihovna nevyužívá\cite{geometryjs:source:procedures:procedure.ts}, ale v budoucnu by mohla být užitečná.

\section[Dokumentace]{Dokumentace procedur}
\label{sec:procedure-documentation}

Jelikož jádrem procedur budou většinou matematické výpočty, jejichž kódová implementacenemusí být dostatečně deskriptivní, je důležité, aby procedury byly dobře zdokumentované.
K tomu máme dedikovanou stránku dokumentace na wiki\cite{geometryjs:wiki:procedures}.

\section[Typování]{Typování procedur}
\label{sec:procedure-typing}

Jelikož procedury dědí z jedné třídy, ale každá procedura má jiný typ vstupního a výstupního objektu, je nutné využít generických typů\cite{TypeScript:generics}.
Vstupní a výstupní typ procedury jsou definovány jako generické typy rozhraní \texttt{Procedure}\cite[line 6]{geometryjs:source:interfaces:procedure.ts}.

\section[Dělení]{Dělení procedur}
\label{subsec:procedure-categorization}

Procedury dělíme do dvou kategorií - procedury \uv{základní} a procedury \uv{odvozené}\cite{geometryjs:wiki:procedures}.

\subsection[Základní]{Základní procedury}
\label{subsubsec:basic-procedures}

Základními procedurami jsou procedury, které nepotřebují specifickou dokumentaci.
Jejich cíl a postup je běžně známý nebo často popisovaný v jiných zdrojích a stačí nám k tomu pouze odkaz na tyto zdroje.

Příkladem může být procedura \texttt{VectorAddition}\cite[line 9-23]{geometryjs:source:procedures:vectorOperations.ts}, která sečte dva vektory, nebo procedura \texttt{VectorDotProduct}\cite[line 88-99]{geometryjs:source:procedures:vectorOperations.ts}, která spočítá skalární součin dvou vektorů.

\subsection[Odvozené]{Odvozené procedury}
\label{subsubsec:derived-procedures}

Odvozené procedury jsou procedury, které potřebují specifickou dokumentaci\cite{geometryjs:wiki:procedures}.
Tyto procedury jsou specifičtější a často například nemají vlastní název.
Jako jejich název použijeme několik slov vystihující, co procedura dělá.

Příkladem může být procedura \texttt{LineLineIntersection}\cite[line 11-27]{geometryjs:source:procedures:lineLine.ts} nebo procedura \texttt{LineCCoefficient}\cite[line 11-20]{geometryjs:source:procedures:lineEquation.ts}.
