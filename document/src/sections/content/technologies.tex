\chapter{Nástroje a jazyk}
\label{chap:technologies}

\section{Programovací jazyk}
\label{sec:programming-language}

Pro tvorbu knihovny jsme si vybrali jazyk TypeScript. 
Jedná se o nadstavbu jazyka JavaScript, která přidává statické typování a další vylepšení\cite{TypeScript:handbook}. 
Jazyk je svou strukturou velmi podobný jazyku C\#.
TypeScript je kompilován do JavaScriptu\cite{TypeScript}, který je následně spouštěn v prohlížeči, zároveň zajišťuje lepší kompatibilitu napříč prohlížeči\footnote{Kompatibilitu zajišťuje tím, že klíčová slova a API, která jsou dostupná až v novějších vezích JavaScriptu, transpiluje do starších verzí.}\cite{TypeScript:tsconfig}.

\section{Nástroje}
\label{sec:tools}

Při tvorbě kvalitní a stabilní knihovny je v podstatě nezbytné využít standardních nástrojů pro vývoj softwaru. 
Tato kapitola se bude zabývat nástroji, které jsme použili při tvorbě knihovny.

\subsection{Verzovací systém}
\label{subsec:version-control}

Verzovací systém je nástroj, který slouží k evidenci změn v kódu. 
Umožňuje nám také vytvářet větve, které můžeme následně sloučit do hlavní větve\cite{Git:branches,wikipedia:version-control}. 
Větve nám umožňují pracovat na více funkcionalitách zároveň, aniž bychom museli mít všechny hotové. 
Také nám umožňuje v případě potřeby se vrátit k předchozí verzi kódu.
 Pokud nalezneme chybu, můžeme v historii najít, jak chyba vznikla podniknout kroky k eliminaci podobných chyb v budoucnu.

Pro tuto práci jsme zvolili verzovací systém Git. Jedná se o nejpoužívanější verzovací systém, který je zdarma a open-source\cite{Git}.
Zároveň má podporu prakticky ve všech standardních vývojových nástrojích a existuje mnoho serverů, které vám umožní zdarma hostovat váš repozitář.
Jako hostovací server jsme vybrali GitHub, jelikož se jedná o nejčastěji používaný server pro open-source projekty\cite{GitHub:about}.

\subsection{Správce balíčků}
\label{subsec:package-manager}

Správce balíčků je nástroj, který slouží k instalaci a aktualizaci knihoven a dalších závislostí.
Také slouží k alspoň částečné standardizaci struktury projektu\footnote{Částečnou standardizací je myšleno, že definuje například kompilační kroky a parametry na jednom místě, nebo obsahuje soubor se standardizovaným názvem, který ve strojově i přirozeně čitelném formátu (typicky YAML nebo JSON) poskytuje informace o daném projektu.}. 
U moderních jazyků je běžné, že většina knihoven je dostupná přes správce balíčků, který je součástí standardní instalace jazyka\footnote{Nebo je ho možno velmi jednoduše přidat.}\cite{npm:docs,Rust:cargo,golang:pkg,python:pip}.

V případě TypeScriptu (ve finále spíše JavaScriptu) je toto poněkud složitější, jelikož nemá standardní instalaci a běžně se spouští přímo v prohlížeči.
Existují však i neprohlížečová prostředí, která umožňují spouštět JavaScriptový kód přímo v příkazové řádce\footnote{Spouští ho prostřednictvím interpreteru v prostředí uzpůsobném pro fungování mimo prohlížeč.}. 
Jedním takovým prostředím je Node.js, které je založeno na jádře V8\cite{node:v8}, které používá i prohlížeč Google Chrome\cite{v8}.
Součástí Node.js je také správce balíčků npm\cite{npm:docs}, který je nejčastěji používaným správcem balíčků pro JavaScript a TypeScript. 
Existuje k němu i pro veřejné balíčky bezplatný repozitář, který je dostupný na adrese \url{https://www.npmjs.com/}.
Tento repozitář je používán i jinými správci balíčků, například pnpm\cite{pnpm}.

\subsection{Kompilátor}
\label{subsec:compiler}

Důležitým \uv{build stepem} knihovny je kompilace. 
Kompilujeme zdrojový kód v TypeScriptu do JavaScriptu, který je spustitelný v prohlížeči i v Node.js. 
TypeScript je, obdobně jako C\#, jazyk vytvořený Microsoftem\cite{TypeScript}.
Microsoft k jazyku tak poskytuje i kompilátor.

Existují i jiné kompilátory, které jsou v některých ohledech\footnote{Například umí efektivněji paralelizovat kompilaci nebo využívat cache.} lepší, ale vzhledem k tomu, že je kompilátor od Microsoftu nejrozšířenější, rozhodli jsme se ho použít.

\subsection{Testování}
\label{subsec:testing}

Automatické testování je důležité pro zajištění kvality kódu v každém projektu. 
U knihovnen je důkladné testování celého kódu ještě důležitější, jelikož se jedná o kód, na který se spoléhají jiné projekty.

Pro testování jsme zvolili framework Jest.
Důvodem byly především naše předchozí zkušenosti s tímto frameworkem.
Protože má všechny potřebné funkce a je velmi jednoduchý na použití, neviděli jsme důvod proč zvolit jiný framework.

\subsection{Dokumentace}
\label{subsec:documentation}

Dokumentace je důležitá pro každý projekt.
Dokumentace knihovny se dělí na dvě části.

\subsubsection{Vývojářská dokumentace}
\label{subsubsec:developer-documentation}

Vývojářská dokumentace je určena pro hlubší pochopení kódu knihovny.
Hodí se například pokud budeme chtít knihovnu rozšířit o nové funkce či hledat zdroje chyb.
GitHub má zabudovanou podporu pro vývojářskou dokumentaci\cite{GitHub:wikis}, která je psaná ve formátu Markdown\footnote{GitHub wikis podporuje i jiné formáty než Markdown, jsou ale méně používané.}.
Markdown je jednoduchý formát, který umožňuje psát text s krátkými formátovacími značkami\cite{GitHub:markdown}.
Vývojářská dokumentace je dostupná na adrese \url{https://github.com/geometryjs/geometry.js/wiki/} a je manuálně aktualizována. 
Jedná se také o hlavní zdroj této práce.

\subsubsection{Uživatelská dokumentace}
\label{subsubsec:user-documentation}

Uživatelská dokumentace (nebo také API dokumentace) je určena pro uživatele knihovny, tedy pro programátory, kteří knihovnu používají.
Ta vychází přímo z kódu knihovny a komentářů v něm.
Je automaticky generována systémem TypeDoc\footnote{Spouštění generování je zajištno přes GitHub Action, která aktivuje \uv{rebuild} stránky na Cloudflare Pages\cite{geometryjs:wiki:api}.} a je dostupná na adrese \url{https://geometryjs.jiricekcz.dev/api/}.
Obashuje soupis všech tříd, funkcí a proměnných, které jsou dostupné pro uživatele knihovny a jejich popis.