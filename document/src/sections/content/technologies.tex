\chapter{Nástroje a jazyk}
\label{ch:technologies}

\section{Programovací jazyk}
\label{sec:programming-language}

Pro tvorbu knihovny jsme si vybrali jazyk TypeScript. 
Jedná se o nadstavbu jazyka JavaScript, která přidává statické typování a další vylepšení. 
Jazyk je svou strukturou velmi podobný jazyku C\#.
TypeScript je kompilován do JavaScriptu, který je následně spouštěn v prohlížeči, zároveň zajišťuje lepší kompatibilitu napříč prohlížeči.

\section{Nástroje}
\label{sec:tools}

Při tvorbě kvalitní a stabilní knihovny je v podstatě nezbytné využít standardní nástroje pro vývoj softwaru. 
V této kapitole se budu zabývat nástroji, které jsme použili při tvorbě knihovny.

\subsection{Verzovací systém}
\label{subsec:version-control}

Verzovací systém je nástroj, který slouží k evidenci změn v kódu. 
Umožňuje nám také vytvářet větve, které můžeme následně sloučit do hlavní větve. 
Větve nám umožňují pracovat na více funkcionalitách zároveň, aniž bychom museli mít všechny hotové. 
Také nám umožňuje v případě potřeby se vrátit k předchozí verzi kódu.
 Pokud nalezneme chybu, můžeme v historii najít, jak chyba vznikla podniknout kroky k eliminaci podobných chyb v budoucnu.

Pro tuto práci jsme zvolili verzovací systém Git. Jedná se o nejpoužívanější verzovací systém, který je zdarma a open-source.
Zároveň má podporu prakticky ve všech standardních vývojových nástrojích a existuje mnoho serverů, které vám umožní zdarma hostovat váš repozitář.
Jako hostovací server jsme vybrali GitHub, jelikož se jedná o nejčastěji používaný server pro open-source projekty.

\subsection{Správce balíčků}
\label{subsec:package-manager}

Správce balíčků je nástroj, který slouží k instalaci a aktualizaci knihoven a dalších závislostí. 
U moderních jazyků je běžné, že většina knihoven je dostupná přes správce balíčků, který je součástí standardní instalace jazyka.

V případě TypeScriptu (ve finále spíše JavaScriptu) je toto poněkud složitější, jelikož nemá standardní instalaci a běžně se spouští přímo v prohlížeči.
Existují však i neprohlížečová prostředí, která umožňují spouštět JavaScriptový kód přímo v příkazové řádce. 
Jedním takovým prostředím je Node.js, které je založeno na jádře V8, které používá i prohlížeč Google Chrome.
Součástí Node.js je také správce balíčků npm, který je nejčastěji používaným správcem balíčků pro JavaScript a TypeScript. 
Existuje k němu i pro veřejné balíčky bezplatný repozitář, který je dostupný na adrese \url{https://www.npmjs.com/}.
Tento repozitář je používán i jinými správci balíčků, například pnpm.

\subsection{Kompilátor}
\label{subsec:compiler}

Důležitým \uv{build stepem} knihovny je kompilace. 
Kompilujeme zdrojový kód v TypeScriptu do JavaScriptu, který je spustitelný v prohlížeči i v Node.js. 
TypeScript je, obdobně jako C\#, jazyk vytvořený Microsoftem.
Microsoft k jazyku tak poskytuje i kompilátor.

Existují i jiné kompilátory, které jsou v některých ohledech lepší, ale vzhledem k tomu, že je kompilátor od Microsoftu nejrozšířenější, rozhodli jsme se ho použít.

\subsection{Testování}
\label{subsec:testing}

Automatické testování je důležité pro zajištění kvality kódu v každém projektu. 
U knihovnen je důkladné testování celého kódu ještě důležitější, jelikož se jedná o kód, na který se spoléhají jiné projekty.

Pro testování jsme zvolili framework Jest.
Důvodem byly především moje předchozí zkušenosti s tímto frameworkem.
Protože má všechny potřebné funkce a je velmi jednoduchý na použití, neviděli jsme důvod proč zvolit jiný framework.

\subsection{Dokumentace}
\label{subsec:documentation}

Dokumentace je důležitá pro každý projekt.
Dokumentace knihovny se dělí na dvě části.

\subsubsection{Vývojářská dokumentace}
\label{subsubsec:developer-documentation}

Vývojářská dokumentace je určena pro hlubší pochopení kódu knihovny.
Hodí se například pokud budeme chtít knihovnu rozšířit o nové funkce či hledat zdroje chyb.
GitHub má zabudovanou podporu pro vývojářskou dokumentaci, která je psaná ve formátu Markdown.
Markdown je jednoduchý formát, který umožňuje psát text s krátkými formátovacími značkami.
Vývojářská dokumentace je dostupná na adrese \url{https://github.com/geometryjs/geometry.js/wiki/} a je manuálně aktualizována. 
Jedná se také o hlavní zdroj této práce.

\subsubsection{Uživatelská dokumentace}
\label{subsubsec:user-documentation}

Uživatelská dokumentace (nebo také API dokumentace) je určena pro uživatele knihovny, tedy pro programátory, kteří knihovnu používají.
Ta vychází přímo z kódu knihovny a komentářů v něm.
Je automaticky generována systémem TypeDoc a je dostupná na adrese \url{https://geometryjs.jiricekcz.dev/api/}.
Obashuje soupis všech tříd, funkcí a proměnných, které jsou dostupné pro uživatele knihovny a jejich popis.